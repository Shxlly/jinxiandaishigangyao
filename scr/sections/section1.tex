\section{导论}
\begin{enumerate}
	\question[学习“中国近现代史纲要”课程的必要性是什么?]
	\unsolve
	\question[什么是中国近现代史?其主流和本质是什么?]
	\ans[中国的近代史近现代史,是指1840年以来中国的历史。其中1850年鸦片战争爆发到1949年中华人民共和国成立前夕的历史,是中国的近代史;1949年中华人民共和国成立以来的历史,是中国的现代史。\par
就其主流和本质来说,是中国一代又一代的人民群众和仁人志士为救亡图存而英勇奋斗、艰苦探索的历史;是全国各族人民在中国共产党的领导下,进行伟大的艰苦的斗争,经过新民主主义革命,赢得民族独立和人民解放的历史;是全国各族人民在中国共产党的领导下,经过社会主义革命、建设和改革,把一个极度贫弱的旧中国逐步变成一个初步繁荣昌盛、充满生机和活力的社会主义新中国的历史。
]
	\question[联系党的十八以来社会主义现代化建设取得的历史性成就,阐述中国选择社会主义的正确性。]
	\unsolve
\end{enumerate}


