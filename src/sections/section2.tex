\section{资本—帝国主义的入侵与近代中国社会的变化}
\begin{enumerate}
	%%%%%%%%%%%%%%%%%%%%%%%%%%%%%%
	\question[为什么说鸦片战争是中国近代史的起点?]
	\ans[随着中国的大门被西方列强用武力强迫打开,中国的社会性质开始发生质的变化。中国从一个享有完整主权的独立国家,逐步成为半殖民地国家,走上了半殖民地半封建社会的畸形发展的道路。随着社会主要矛盾的变化,中国逐渐开始了反帝反封建的资产阶级民主革命。正因为如此,鸦片战争就成了中国近代史的起点。]

	%%%%%%%%%%%%%%%%%%%%%%%%%%%%%%
	\question[什么是半殖民地半封建社会?中国的半殖民地半封建社会是怎样形成的?]
	\ans[
		鸦片战争以后,西方列强通过发动侵略战争、强迫中国签订一系列不平等条约,破坏了中国的领土主权、领海主权、关税主权、司法主权等,并一步一步地控制了中国的政治、经济、外交和军事。中国已经丧失了完全独立的地位,在相当程度上被殖民地化了。近代中国尽管在实际上已经丧失了拥有完整主权的独立国的地位,但是仍然维持着独立国家和政府的名义与形式。由于它与连名义上的独立也没有、而由殖民主义宗主国直接统治的殖民地尚有区别,因此称作半殖民地。\par
		外国资本主义列强用武力打开了中国的门户,把中国卷入了世界资本主义经济体系和世界市场之中。随着外国资本主义的入侵,洋纱、洋布等商品在中国大量倾销,逐渐使中国的农业与手工业分离,从而破坏了中国自然经济的基础,促进了中国城乡商品经济的发展,给中国资本主义的产生造成了某些客观条件。破产的农民、手工业者流入城市,成了产业工人的后备军。一批中国官僚、买办、地主、商人开始投资兴办新式工业。中国开始出现了资本主义生产关系。中国已经不是完全的封建社会了。然而,西方列强并不愿意中国成为独立的资本主义国家。它们利用获取的政治、经济特权,在中国倾销商品,经营轻工业和重工业,对中国的民族工业进行直接的经济压迫。中国的民族资本主义经济虽然有了某些发展,但是并没有也不可能成为中国社会经济的主要形式。而在中国的资本主义经济中,外国资本及依附于它的官僚资本居于主导地位。在中国农村中,地主剥削农民的封建生产关系,在社会经济生活中依然占着显然的优势。这样,中国的经济既不再是完全的封建经济,也不是完全的资本主义经济,而成为半殖民地半封建的经济了。
	]
	
	%%%%%%%%%%%%%%%%%%%%%%%%%%%%%%
	\question[试述中国梦的由来和内涵。]
	\ans[{
               由来:{
                  1.中华民族五千年文明史,各项成就推动了政治、经济和文化的发展,为人类社会进步做出巨大贡献,中国古代灿烂文明是中国梦之历史源起;
                  2.1840年开始的列强侵华,中国梦是对列强侵华、落后挨打的抗争,历经太平天国、洋务运动个、戊戌变法、辛亥革命,半殖民地半封建社会的性质没有改变,反帝反封建的任务没有完成,民族独立和人民解放、国家富强和人民富裕,始终是那个年代中国人无法实现的梦想;
                  3.中国共产党人前赴后继、不懈奋斗,完成了新民主主义革命,建立了新中国,实现了民族独立、人民解放;完成了社会主义革命,确立了社会主义基本制度,进行了改革开放的的伟大革命,开创、坚持、发展了中国特色社会主主义,从根本上改变了中国人民和中华民族的前途命运,为实现中国梦积累了宝贵经验,、提供例如理论指导、奠定了物质基础;
                             }
               内涵:{
                   国家富强、民族振兴、人民幸福 {
                   1.中国梦是实现中华民族伟大复兴的崇高理想;全面建成小康社会、建成富强民主文明和谐的社会主义现代化国家,实现“两个一百年”的奋斗目标,就是现阶段中国人民的共同愿望;
                   2.中国梦是国家、民族、社会梦想和大众梦想的统一;要把实现国家强盛、民族复兴、社会进步,同中国人民的共同富裕、全面小康结合起来;
                   3.中国梦对于不同阶层、不同群体有不同的愿景和诉求。
                              } ]
	
	%%%%%%%%%%%%%%%%%%%%%%%%%%%%%%
	\question[资本—帝国主义的入侵给中国带来了什么?]
	\ans[
		\begin{enumerate}
		\item{\textbf{军事侵略}\par
			发动侵略战争,屠杀中国人民;\par
			侵占中国领土,划分势力范围;\par
			勒索赔款,抢掠财富;\par
		}
		\item{\textbf{政治控制}\par
			控制中国的内政、外交;\par
			镇压中国人民的反抗;\par
			扶植、收买代理人;\par
		}
		\item{\textbf{经济掠夺}\par
			控制中国的通商口岸;\par
			剥夺中国的关税自主权;\par
			实行商品倾销和资本输出;\par
			操控中国的经济命脉;\par
		}
		\item{\textbf{文化渗透}\par
			披着宗教外衣,进行侵略活动;\par
			为侵略中国制造舆论;\par
		}
		\end{enumerate}
	]
	
	%%%%%%%%%%%%%%%%%%%%%%%%%%%%%%
	\question[1840年至1919年间中国历次反侵略战争失败的原因是什么?有何教训?]
	\ans[
		\textbf{原因:} \begin{itemize}
			\item {社会制度的腐败:1840年以后,中国封建社会逐步变成了半殖民地半封建社会。1911年以前统治中国的清王朝,从皇帝到权贵,大都昏庸愚昧,不了解世界大势,不懂得御敌之策。许多官员贪污腐化,克扣军饷。不少将帅贪生怕死,临阵脱逃。他们大多害怕拥有坚船利炮的外国侵略者,甚至为了自身的私利,不惜出卖国家和民族的利益。他们尤其害怕人民群众,担心人民群众动员起来以后可能危及自身的统治。所以,他们不仅不敢放手发动和依靠人民群众的力量,而且常常压制与破坏人民群众和爱国官兵的反侵略斗争。}
			\item {经济技术的落后:近代中国反侵略战争失败的另一个重要原因,是国家综合实力特别是经济技术和作战能力的落后。19世纪中叶,西方资本主义强国经过工业革命,经济和技术飞速发展,封建的中国已被远远抛在后面。}\end{itemize}
		\textbf{教训:}\begin{itemize}
			\item{“师夷长技以制夷”和早期维新思想。中国官吏和知识分子中的少数爱国、开明的有识之士,开始注意了解国际形势,研究外国史地,总结失败教训,寻找救国的道路和御敌的方法。他们通过收集、翻译传入的外国报刊、书籍、地图,以及战争中审问英军俘虏和向外国商人、传教士直接询问等各种方式,来获取世界知识。}
			\item{救亡图存和振兴中华。鸦片战争以后,中国还只是少数精英开始有了朦胧的民族觉醒意识。到了中日甲午战争以后,当中华民族面临生死存亡的关头时,全民族就开始有了普遍的民族意识的觉醒。}\end{itemize}			
	]
\end{enumerate}
